% !TeX root = ../main.tex

\begin{comments}
% \begin{comments}[name = {指导小组学术评语}]
% \begin{comments}[name = {Comments from Thesis Supervisor}]
% \begin{comments}[name = {Comments from Thesis Supervision Committee}]

强化学习算法是一种高效的智能决策方法。但样本利用效率和策略安全性是其应用于真实场景的关键瓶颈之一。优先经验回放算法通过改变回放样本分布,优先学习部分样本的做法提升了样本利用效率。基于模型的强化学习算法通过对环境建模产生模拟数据辅助学习,也能加快学习速度。两者均是解决样本利用效率的较好思路,但这些方法在提升样本利用效率的同时,无法保证策略的稳定性。

该论文针对以上问题,开展了对强化学习算法的样本利用效率和安全性研究,选题具有重要价值。论文主要取得了以下几个成果: 

\begin{enumerate}
    \item 提出了一种基于双层缓存的优先经验回放算法(Double-layer Prioritized Experience Replay,DPER),通过双层经验回放池的设计,以不同速率进行优先级经验回放,在同等的样本消耗量下,实现了对样本空间更大的经验回放覆盖率,有效提升了强化学习算法的样本利用效率。
    \item 提出了一种基于模型集成的筛选规划算法(Model-Based Dropout Planning,MBDP),通过将集成模型筛选模块和模拟数据筛选模块相整合,以对抗的学习方式,设计了一个可以动态取舍样本利用效率和策略鲁棒性的算法框架,最终实现了在保证样本利用效率的前提下对所学决策策略安全性进行有效的加强。
    \item 在真实温室环境中,将所提出的DPER算法与MBDP算法整合,设计了自动化控制策略学习框架,在严苛的条件下对提出的算法进行了全面充分的验证,对于强化学习算法在真实复杂场景中的可应用性有重要意义。
\end{enumerate}

 论文结构合理,层次清晰,文献详实,实验充分,用词准确,将自己的工作进行了详细的阐述。论文工作表明该同学已掌握本学科坚实的理论基础知识,具有较强学习能力和动手能力,具备独立从事科学研究的能力。论文达到了工学硕士学位论文的要求,同意其答辩申请,并建议安排答辩。

\end{comments}
