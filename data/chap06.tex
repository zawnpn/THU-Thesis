% !TeX root = ../main.tex

\chapter{总结与展望}

\section{全文总结}

在智能决策学习任务中,算法的样本利用效率和所学策略的安全性往往限制了强化学习的实际应用效果,样本利用效率低下会带来高昂的采样成本,安全性的不足会导致策略做出不稳定甚至错误的决策。针对智能决策学习任务中的这些问题,本文首先提出基于双层缓存的优先经验回放算法(Double-layer Prioritized Experience Replay,DPER)和基于模型集成的筛选规划算法(Model-Based Dropout Planning,MBDP)。DPER算法通过在传统优先经验回放算法(PER)的基础上添加了第二层经验回放池,用于长时缓存全局空间的经验回放样本,达到进一步增加强化学习算法的样本利用效率的效果。而MBDP算法则以对抗的训练方式,设计了能够提升样本利用效率的集成模型筛选模块,以及能够提升算法鲁棒性的模拟数据筛选模块,两者结合而成的MBDP算法能够在保证算法样本利用效率的前提下,对鲁棒性起到大幅的提升作用。

本文首先在CliffWalking和OpenAI提供Mujoco环境下进行了模拟仿真实验,对所设计的DPER算法和MBDP算法进行了实验验证。实验表明,DPER算法中所提出的双层经验回放池设计能够有效地增强策略学习的收敛速度,相比优先经验算法(PER)有着更好的样本效率。而MBDP算法也在Mujoco环境下的验证实验中表现出了优秀的鲁棒性,即使环境发生干扰,也能一致地表现出收敛性能,验证了MBDP算法能够在保证算法样本利用率的同时,起到鲁棒性的提升效果这一结论。

为了进一步验证所提出的算法在真实应用场景中的提升效果,本文以基于模型集成的筛选规划算法为核心,针对自动化温室决策控制任务,设计了能够在温室中根据传感器信息自动控制设备的